%%%%%%%%%%%%%%%%%%%
%Research Survey%%
%%%%%%%%%%%%%%%%%%%

\chapter{Research Survey}\label{chap_r}

\section{Introduction}\label{sec_r1}
Market microstructure is the study of the process and outcomes of exchanging assets among financial intermediations and institutions of exchange.  As is suggested by its name, the study of market microstructure has been developed along with the expansion of practical business needs under explicit trading rules and objectives.  Although the study of market microstructure spreads over various issues from macroscopic to microscopic as well as from theoretical to empirical, we would like to illustrates two large issues, market design in Section \ref{sec_r2} and optimal strategy in Section \ref{sec_r3}, which are most relevant to this study.

Looking at the history of investment business and research, we can see several stages.  First, the modern portfolio theory has been established in 1960's and developed in 1970's.  The greatest academic achievements include studies on the long-term equilibrium of financial markets, such as Portfolio Selection by Markoviz (1959), Capital Asset Pricing Model by Sharpe (1970), and Arbitrage Pricing Theory by Ross (1976).  These concepts provided great progress on the investment management, and as a consequence, portfolio investment was introduced and gradually took over investment on individual securities.  However, this is also the start of the long lasting argument on the gap between academic research and practical business partly because the practical market is sometimes far away from the equilibrium.

Since the deregulation of the Securities markets such as May Day in 1975 in the U.S.A. and Big Bang in 1986 in the U.K., trading activity has burgeoned.  This has led to a widening of the gap between the ideal equilibrium market and the practical market to the extent that it can no longer be considered negligible.  Consequently, the need for efficient market design was inevitable, and this gave birth to the study of market microstructure; an analysis of the dynamics of price formation.


%%%%%%%%%%%%%%%%%%%%%%%%
\section{Market Design}\label{sec_r2}
In this study area, empirical studies preceded theoretical studies in order to verify practitioners' feelings that the real market could not be explained by existing finance theories.  Then, theoretical studies of general equilibrium followed to provide descriptive explanations to these empirical findings.

Regardless of empirical and theoretical, the main parameters in market microstructure are price and quantity, just as the existing studies on price formation in micro economics.  Especially, market microstructure analyzes dynamical mechanisms, these parameters often appears as the change of price or volatility, and trading volume flow, which is henceforth referred as trading volume for simplicity.  

Due to the restriction of data and analytical tools, earlier studies observe these two parameters separately in the form of volatility and trading volume, which we call studies on market pattern and explain in Subsection \ref{subsec_r21}.  Later studies tied individual trades and price changes, and observe market impact, which are reflected in Subsection \ref{subsec_r22}.  Further, most of the existing studies are on quote driven markets because market data on order driven markets is too complex.  However, we pick up some among a few papers on order driven markets in Subsection \ref{subsec_r23} since the understanding of quote driven market is necessary for Chapter \ref{chap_l} of this study.


\subsection{Market Pattern}\label{subsec_r21}
%from(2.3.1)
It is known that some patterns exist in actual price and liquidity, which cannot be explained by existing finance theories.  For example, Jain and Joh (1988) analyze trading data of NYSE stocks from 1979 to 1983 and report the ``U-shaped effect" that trading volume is concentrated at the opening and closing of trading sessions, that trading volume on Monday is lower than on other days of the week while weekday pattern is smaller than intraday pattern, and that trading volume is closely related to price movement four hours before, which suggests effect of price movement lasts for a while.

Further, McInish and Wood (1992) analyze the intraday pattern of quote spread on the NYSE.  They find that quote spread tightens when trading is active.  Also, quote spread is larger for larger orders and smaller for more competitive markets to market makers.  While they also confirm the ``U-shaped effect" in the intraday pattern at NYSE, quote spread in a quote driven market such as NASDAQ and London Stock Exchange (LSE) remains large throughout the day, and tightens rapidly towards the closing.  Therefore, they conclude that difference in trading mechanism causes intraday pattern of quote spread.

Then, Chan et al. (1995) study quote spread of individual market makers and inside spread (difference between best quotes) for NASDAQ stocks.  They find that individual quote spread remains stable throughout the day and reveal that decline of inside spread at closing is due to heterogeneity of quote spread among market makers to unwind their positions.  Further, Barclay et al. (1999) find that after NASDAQ reform in which public traders became able to submit limit orders, NASDAQ shows the ``U-shaped effect" as NYSE.

Numerous studies have been done on the relationship between price volatility and market trading volume since Clark's (1973) seminal paper, including Epps and Epps (1976), Tauchen and Pitts (1983).  Karpoff (1987) summarizes these results, and Andersen (1996) proposes a modified model.  These studies, in general, support the existence of a positive correlation between price volatility and market trading volume and the autocorrelation of themselves because surprising news boosts both price volatility and market trading volume.  

Regarding Japanese markets, Kawahara and Murase (1993) analyze intraday data of TSE stocks in 1993.  They attribute factors of price movement within a day into quote spread, execution period, and trading volume.  

On the theory side,
%from(2.4-2.4.1)
 trade models are devised in order to provide customary market designs and patterns with theoretical grounds.  Market participants in the real world do not always have all the public information and behave rationally, but act with different information, objectives, and circumstances.  Therefore, real markets are too complex to analyze.  As a result, existing trade models employ significantly simplified assumptions on the roles and objectives of market participants, and try to obtain qualitative implications.  

%\subsection{Strategic Trader Model}
One of the pioneering works is done by Kyle (1985), who assumed that traders submit market orders, and then market makers settle the surplus of traders' orders and determine the price.  According to this model, the excess demand of the informed trader is larger when volatility of settlement price is lower and when uncertainty of noise trader is higher.  Also, quote spread is smaller when the volatility of settlement price is lower and when uncertainty of the noise trader is higher.  Further, this study specifies the abstract idea of market liquidity as 1) existence of quote, 2) small quote spread, 3) capability of successive trades, and 4) capability of instantaneous execution of large trade.  Liquid market thus represents an efficient market in the sense that orders of any size are tradable instantaneously around market prices.

Admati and Pfeiderer (1988) analyze a multiple period model in order to explain the ``U-shaped effect."  They show that orders by uninformed traders tend to cluster in the period with more informed traders such as opening and closing, and price volatility also increases.  

Subrahmanyam (1991) assumes riskaverse informed traders and finds that informed traders reduce orders to avoid risks from noise traders, and therefore, price does not reflect information effectively.

%%%%%%%%%%%%%%%%%
\subsection{Market Impact}\label{subsec_r22}
%from(2.3.2)
Market impact measures the extent to which short-term liquidity makes execution prices diverge from long-term fundamental prices. Holthausen et al. (1987) analyse block trades, and find that prices move in the opposite direction after block trades.  They conclude that block orders not only introduce new information to adjust price levels but also bring temporary price turbulence due to low liquidity and speculations.  As a result, they distinguish temporary impact and permanent impact.  

Also, Hausman et al. (1992) focuses on discreteness of execution price and estimate market impact precisely with an ordered probit model.  Further, BARRA (1997) develops a practical market impact model whose impact is an increasing function of trade size and the marginal increase is a decreasing function.  BARRA's model attributes market impact to individual factors (price elasticity of quote, volatility, number of ticks, and trading size) and market factors.

Since analysis on market impact was born in the United States, where quote driven markets are prevailing, there are few studies on order driven markets.  It is difficult to identify block orders from trading data in an order driven market since trading data is merely a sequence of times, prices and volumes of ticks which are split into single limit orders on the limit order book. In addition trading data does not show lumps or the direction of orders.

Uno and Yamada (1993) analyze intraday data of TSE stocks.  They regard down-ticks as sell orders and up-ticks as buy orders and aggregate net buy orders into five-minute intervals.  They explained market impact is determined by long-term characteristics of the asset such as average trading volume and volatility and short-term characteristics of trading such as tick trend, trading volume, and the depth of the limit order book.

\subsection{Limit Order}\label{subsec_r23}
Most of the existing studies are on quote driven markets because market data in order driven markets is too complex.  However, we pick up some among few papers on order driven markets in this subsection since the understanding of quote driven market is necessary for Chapter \ref{chap_l} of this study.

%from(4.1)
Market and limit orders are two essential instruments of order placement.  Progress in information technology has improved order processing capability and made order placement strategy a critical issue in investment.  For example, order driven markets such as the Tokyo Stock Exchange (TSE) have excluded floor traders and adopted automatic matching systems, which makes order processing faster and less error prone.  Also, several markets such as New York Stock Exchange (NYSE) and National Association of Securities Dealers Automated Quotation System (NASDAQ) started allowing limit orders, to broadening traders' order placement options.  Further, emergence of alternative trading systems such as crossing networks and electronic communication networks have enabled traders to tactically submit and cancel orders across markets.  In practice, traders recognize the advantage of limit orders: cost efficiency, and disadvantages: uncertain execution and free option, and choose either market or limit order or sometimes both as it fits their objectives and circumstances.  

Several empirical studies confirm traders' behavior in the real market.  For example, 
%from(2.3.3)
Biais et al. (1995) analyze trading data of Paris Stock Exchange, an order driven market, and find that traders are affected by limit order books and trades just before the order submission.  Especially, more market orders are submitted when the bid-ask spread is small, and more limit orders are submitted inside the bid-ask spread when the bid-ask spread is large, which suggests that traders sufficiently consider the execution probability of limit orders.

Harris and Hasbrouck (1996) analyze data of NYSE and calculate that execution probability of limit orders within the bid-ask spread is 60\%.  They show that trading cost of limit orders is lower than other strategies when bid-ask spread is wide, which is consistent with the result of Biais et al. (1995).

Regarding the Japanese market, Kawahara (1994) analyzes TSE stock data in 1994 and finds that the trading cost of limit orders outside the best bid and ask prices is 0.15\% lower than that of market orders.  Also, that the trading cost is larger for sell orders, large orders, and limit orders with shorter exposure to the market.

In contrast, theoretical studies are limited for the selection of market and limit orders.  For example, although Bertisimas and Lo (1998) and Chapter \ref{chap_b} of this study derive the optimal order slicing strategy in a large portfolio liquidation, only market orders but no limit orders are allowed in their models.  Also, Parlour (1999) analyses how market conditions affect the selection of market and limit orders when a trader places just one unit of order.  This analysis shows the interaction between trader's order placing strategy and market conditions, but fails to study how traders split orders for large trades.  Further, Chakravarty and Holden (1995) provide a theoretical model which explains how traders select market and limit orders in a quote driven market.  However, existence of a market maker is crucial in their model, and the results are not directly applicable to an order driven market.  Accordingly, Chapter \ref{chap_l} of this study analyzes the optimal selection of market and limit orders, sizes, prices, and times in a series of single price batch auctions.  

\subsection{Closing Remarks}\label{subsec_r24}
%from(2.3.4)
Empirical studies successfully reveal practical market anomalies.  Especially, it is shown that market and limit orders should be chosen depending on market circumstances and traders' objectives although these orders should be indifferent theoretically in an efficient market.  However, analysis on order driven markets has just been started, and numerous issues are left for further analysis since the trading data is complex and disclosure has not been sufficient and also since most exchanges in the United States where research on market microstructure was born are quote driven market.

%from(2.4.4)
On the theory side, trade models successfully reveal the roles of market participants and show the conditions in which market stability is retained.  However, the models are significantly simplified in order to avoid complexity, and therefore, reflection of the practical market is left for succeeding researches of the optimal strategy, which are explained in Section \ref{sec_r3}.

%%%%%%%%%%%%%%%%%%%
\section{Optimal Strategy}\label{sec_r3}
In the 1990's, world equity market parcipitants prosperedwhile market prices trended higher.  These conditions favoured the birth of numerous hedge funds.  Investment business has become more competitive, and trade execution has attracted more attention.  With the deep understanding of market mechanisms brought by previous studies on market designs, the study of market microstructure has been revealed to be another large issue of the optimal strategy.
Whereas theoretical models in market design tend to use qualitative analysis and general equilibrium, theories in optimal strategy lend themselves more towards quantitative analysis and the partial equilibrium approach.

The first major task is the conceptual definition of trading cost by
Perold (1988), which is explained in Subsection \ref{subsec_r31}.  With this trading cost in mind, earlier theoretical studies try to determine the optimal portfolio with specific forecast, which is explained in Subsection \ref{subsec_r32}.  After trade execution is recognized as an independent function, we saw several studies on the optimal execution as explained in Subsection \ref{subsec_r33}.

\subsection{Trading Cost}\label{subsec_r31}
%from(1)
According to the ``implementation shortfall method" by Perold (1988), the standard framework of market microstructure analysis, trading costs are defined as difference between the market price at the time of decision-making and evaluating price. The evaluating price consists of the executed price of the filled order and the evaluating price of the unfilled order.  Further, trading costs are decomposed into four components, fixed commisions, timing cost, market impact, and opportunity cost as we see in Figure \ref{fg_i0}.  Among them, timing cost is the difference between market price at the time of decision-making and time of order placement. Market impact is the difference between market price at the time of order placement and the executed price, which is caused by the relationship between trading volume and price change. Lastly opportunity cost is the difference between the executed price and the evaluating price of unfilled order.  Strictly speaking, the nature of timing cost and opportunity cost is the price movement risk but is not necessarily a loss.  However, the price movement risk is often recognized as cost after being mulitiplied with a conversion factor, a risk premium, because most traders are risk averse.  Trade execution strategy can control market impact and price movement risk.

\subsection{Optimal Portfolio Selection}\label{subsec_r32}
%from(2.5.1)
Konno and Wijayanayake (1998) develop a mean absolute difference model in order to build optimal rebalance portfolio under a non-linear trading cost function.  They take into account trading costs for rebalance (difference between existing and optimized portfolio) in addition to expected return and risk as usual. Generally, average trading cost is large for small orders, and decreases as trade size increases.  However, over large trades tighten order supplies and increase trading cost again.  As a result, average trading cost is a ``reverse S-shaped" function of trade size.

In order to solve this complicated optimization numerically, this study measures price movement risk in absolute difference and approximates trading cost as a piecewise linear function, which simplifies the problem into piecewise linear programming.

\subsection{Optimal Execution}\label{subsec_r33}
%from(3.1)
In various aspects of financial decision-making from risk management to trading, it is commonly assumed that any trade can be executed around a prevailing price within a short period of time.  However in practice, we sometimes face situations in which market cannot fully absorb all the trading needs of a large portfolio.  In such cases, asset liquidity risk, defined as the potential deviation between market price and executed price, becomes significant.  Liquidity risk has attracted much attention since the LTCM 1998 disaster, as described in Jorion (2000).  Conventional VaR measures should be extended to account for liquidity risk as well as optimal trading strategies.  This leads to the concept of L-VaR, a liquidity-adjusted risk measure. 

Since asset liquidity risk is a highly complicated problem, there used to be a large gap between practice and theory.  On the practice side, it is known that in liquidation of a block trade portfolio, traders generally divide the whole trade into small orders in order to reduce the cost.  In doing so, it is common to take a longer time for larger trades.  Also, BARRA (1997)'s analysis on quote driven market and Mannen and Uno (2000)'s analysis on order driven market show that the quoted premium of a block trade in the real market is an increasing function of whole trading size, and that the marginal increment is a decreasing function (i.e., the first derivative is positive and the second derivative is negative).  

On the theory side, Collins and Fabozzi (1991) proposed a conceptual framework, which decomposes trading cost into execution cost and opportunity cost.  According to their research, quick execution suffers huge execution cost, while slow execution creates large opportunity cost from price movement risk.  Therefore a trader chooses the best strategy that balances these two factors in determining execution schedule.

Subsequent studies investigated more detailed theory.  Bertsimas and Lo (1998) derived the optimal execution strategy of a large block of common stock over a fixed time horizon by dynamic programming.  However, they analyzed only risk neutral traders because of computational complexity of dynamic programming.  So, they just minimized the expectation value of execution cost, but did not consider price movement risk, and therefore, no reasonable guidance on the duration of execution was provided.  On the other hand, Grinold and Kahn (1999) and Almgren and Chriss (1999) took the sum of execution cost and opportunity cost of the whole trade, as the objective function to minimize, and derived a static optimal execution strategy.  Although they studied risk averse traders, optimal execution duration and average cost are still independent of whole trading size, which contradicts the result of BARRA (1997) and Mannen and Uno (2000).  Further, Grinold and Kahn (1999) contradicts their own statement that risk increases as square root of time, since their risk measured in variance, in fact, increases linearly.  This is crucial because the time dependency of cost and risk determines the optimal execution strategy.

To the contrary, in order to bridge the gap between theory and practice, this chapter modifies objective function in a practical manner, which reflects the reality of the order driven market, and obtains a reasonable relationship between transaction cost and whole trading size.  Specifically, we extend the results on optimal execution strategy derived by Grinold and Kahn (1999) and Almgren and Chriss (1999).  As mentioned above, the explicit solutions obtained in the earlier work have the unrealistic properties in that the time scale of optimal execution strategy is independent of whole trading size, and the total cost of trading is linear in whole trading size. 

There are two possible ways to capture these effects in a quantitative model. One way is to modify the cost functions to be nonlinear.  Another way is to change the weighting of cost uncertainty from a mean-variance model based on a smooth utility function, to a penalty based on standard deviation, which is more in the spirit of VaR models.  This approach was briefly discussed in Almgren and Chriss (1999) where it was observed that solutions could be obtained graphically on the efficient frontier, but explicit solutions were not obtained due to the nonlinearity of the problem, nor were the properties of these solutions explored.

For the standard deviation model, our study in Chapter \ref{chap_b} derives explicit solutions of optimal execution strategy under asset liquidity risk.  It extends the work of Almgren and Chriss (1999), who derive a measure of L-VaR.  We obtain explicit solutions both for the single-asset case and for a multi-asset portfolio model in the form of pure exponentials.  The explicit form obtained for the coefficients shows that the characteristic time increases as whole trading size to the two-thirds power, and total cost increases as whole trading size to the four-thirds power, both consistent with intuition.

%from(6.1)
Optimal execution is very important in both fixed-price and VWAP trades.  It is true that VWAP has become an industry standard, and that numerous studies have been done on market pattern as we have seen in Subsection \ref{subsec_r21}, and that several studies have been made on optimal strategies for general trade execution.  However, little academic work can be found on the optimal execution of VWAP trades.  Therefore, Chapter \ref{chap_s} and \ref{chap_d} analyze the static and dynamic optimal execution of VWAP trades, respectively.

\subsection{Closing Remarks}\label{subsec_r34}
%from(2.5.3)
Previous studies successfully balance market friction and traders' objectives, which is not addressed by existing finance theory.  However, for analytical feasibility, most studies are too simplified and sometimes inconsistent with market practice.  Further, there are several trading methods in practice.  Policies which suggests how to select appropriate trading method according to asset characteristics and traders' objectives and circumstances are left for further analysis.
%from(1.3)
In practice, all the security trading is carried out through some trade execution, regardless of recognition.  At execution, one of the four approaches above is chosen according to assets' characteristics and traders' objectives and circumstances.  In order to make an appropriate choice, traders have to deeply understand the nature of the alternatives.  

Although few analyses have been made about the optimal trade execution based on findings of market microstructure, this study derives the optimal execution strategy in each approach, analyzes its characteristics, and provides the guidance to the selection of these approaches.  Therefore, an extensive range of application can be expected for any security trading.
